\documentclass[12pt,twoside]{article}

\newcommand{\reporttitle}{ROW -- Business Case}
\newcommand{\reportauthor}{ROW ROW ROW YOUR BOAT 5}
\newcommand{\reporttype}{Rescue On Wheels}
\newcommand{\cid}{your college-id number}
\newcommand{\teamnames}{Martijn Vegter (500775388)\\
                        Nino van Galen (500790589)\\
                        Christiaan van Arum (500778983)\\
                        Rapha\"el Bunck (500774349)}

% include files that load packages and define macros
\usepackage{graphicx}
\usepackage{booktabs}
\usepackage[english]{babel}

\hypersetup{pdfpagemode=FullScreen}
 % various packages needed for maths etc.
\input{notation} % short-hand notation and macros

\usepackage[fencedCode,inlineFootnotes,citations,definitionLists,hashEnumerators,smartEllipses,hybrid]{markdown}

%%%%%%%%%%%%%%%%%%%%%%%%%%%%

\begin{document}
% front page
\title{Business Case}
\author{ROW 5}
\date{\today}

\makeatletter
\let\thetitle\@title
\let\theauthor\@author
\let\thedate\@date
\makeatother

\pagestyle{fancy}
\fancyhf{}
\rhead{\theauthor}
\lhead{\thetitle}
\cfoot{\thepage}


	\begin{titlepage}
		\centering
		\vspace*{0.5 cm}
		\includegraphics[scale=1]{HvA.jpg}\\[1.0 cm]
		\textsc{\Large Rover Rescue System}\\[0.5 cm]
		\rule{\linewidth}{0.2 mm} \\[0.4 cm]
		{ \huge \bfseries \thetitle}\\
		\rule{\linewidth}{0.2 mm} \\[1.5 cm]
		
		\begin{minipage}{0.4\textwidth}
			\begin{flushleft} \large
				\emph{Author:}\\
				Christiaan van Arum\\
				Rapha\"el Bunck\\
				Nino van Galen\\
				Martijn Vegter
			\end{flushleft}
		\end{minipage}~
		\begin{minipage}{0.4\textwidth}
			\begin{flushright} \large
				\emph{Student Number:} \\
				500778983\\ % Chris 
				500774349\\ % Raphaël
				500790589\\ % Nino
				500775388	% Martijn
			\end{flushright}
		\end{minipage}\\[2 cm]
		
		{\large \thedate}\\[2 cm]
		
		\vfill
		
	\end{titlepage}



%%%%%%%%%%%%%%%%%%%%%%%%%%%% Main document
\tableofcontents
\newpage
\begin{markdown}

## Description
> Technology has become the answer to a lot of today's problems. Transportation, communication and security are a few examples where IT has come in handy. When you combine these possibilities and take a great team to develop a product, you get a rover that can help humans in tight positions. To realize a product that can rescue humans in disasters such as earthquake-sites, terrorist attacks and the like, the project ``Rescue On Wheels'' has been setup.

## Problem, need and advantages

### Problem
> The moment an earthquake happens or a bomb goes off the location is considered dangerous. On said site other people should not be put in danger too. Nevertheless the places where victims might be stuck could be not easily reachable by humans. And even if the location is easily accessible, another disaster could be triggered by a heavy object such as a human.

### Need
> Given the previous stated problem the idea of a robot. The robot we will design will be a semi-remote-controlled unit the size of a big cat. The rover will be able to explore a disaster-site with ease due to its small size. In addition to its size we'll design the rover with a special treat, being a pan-tilt camera.

### Advantages
> The fact that that a disposable robot can be put in danger compared to a person is a great advantage. The size --and thus the weight-- of the robot will be a factor that makes rescues a lot more likely to succeed. Additionally the pan-tilt camera module enables the operator to use virtual-reality to emerge in the site.
\newpage

##Schedule
### Sprint 1
> In the first sprint of this project we will focus on the basic operations of the rover. This means that we can move the rover forwards, backwards and make it turn.

### Sprint 2
> In the second sprint we will add the functionality of remote control via a mobile application. This will be done via Wi-Fi, radio or 4g.

### Sprint 3
> This sprint will be dedicated to the the ability of streaming the camera feed to the control application.

### Sprint 4
> In the last sprint we will focus on adding image recognition.

\newpage
## Requirements
### Functional
\begin{itemize}
  \item The robot must be able to drive forward, backwards and be able to make turns.
  \item The robot can be operated form an other location, by means of a controller
  \item The robot will be able to stream its camera sight.
  \item The robot will be able to recognize people.
\end{itemize}
    
### Non-functional
\begin{itemize}
    \item The user can look around with the VR-headset.
    \item The camera mount mimic the position of the VR-headset.
    \item The rover will auto-stop before collition.
\end{itemize}

\end{markdown}

\end{document}