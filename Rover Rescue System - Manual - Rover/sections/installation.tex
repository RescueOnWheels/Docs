\section{Installation Guide}
This chapter gives a step by step tutorial to get the RSS up and running. After completing the installation, continue reading the usage guide.

\subsection{Pre-installation}
Before installing the Rover software, make sure you have the following software up and running:
\begin{itemize}
    \item A RaspberryPi B2 (RPI) with newest version of Raspbian.
    \item An Active internet connection on the RPI.
    \item An open SSH, see setup below.
\end{itemize}

\subsection{Headless setup procedure}

\begin{itemize}
    \item Step 1: Create an empty file. You can use Notepad on Windows or TextEdit to do so by creating a new file. Just name the file ssh. Save that empty file and dump it into boot partition (microSD).
    \item Step 2: Create another file name \textbf{wpa\_supplicant.conf}. This time you need to write a few lines of text for this file. For this file, you need to use the \textbf{FULL VERSION} of wpa\_supplicant.conf. Meaning you must have the 3 lines of data namely country, \textbf{ctrl\_interface} and \textbf{update\_config}.
\end{itemize}

\begin{lstlisting}
}
country="Your country"
ctrl_interface=DIR=/var/run/wpa_supplicant GROUP=netdev
update_config=1

network={
    ssid="your_real_wifi_ssid"
    scan_ssid=1
    psk="your_real_password"
    key_mgmt=WPA-PSK
}
\end{lstlisting}
\newpage
\subsection{Installation}
\noindent To install Rover and its dependencies from your RPI shell prompt, use the following command:
\begin{lstlisting}[language=bash]
  $ wget -q https://git.io/fx0ko -O /tmp/rover && bash /tmp/rover
\end{lstlisting}

\subsection{Manual Installation}

Update your system package list:
\begin{lstlisting}[language=bash]
  $ sudo apt-get -update -y
\end{lstlisting}

Upgrade all you installed packages to their latest version:
\begin{lstlisting}[language=bash]
  $ sudo apt-get dist-upgrade -y
\end{lstlisting}

To download and install newest version of Node.js, use the following command:
\begin{lstlisting}[language=bash]
  $ curl -sL https://deb.nodesource.com/setup_8.x | sudo -E bash -
\end{lstlisting}

Now install it by running: 
\begin{lstlisting}[language=bash]
  $ sudo apt-get install nodejs -y
\end{lstlisting}

To compile and install native addons from npm you also need to install build tools: 
\begin{lstlisting}[language=bash]
  $ sudo apt-get install build-essential -y
\end{lstlisting}

To download the Rover code you need to install git: 
\begin{lstlisting}[language=bash]
  $ sudo apt-get install git -y
\end{lstlisting}

Clone 'Scriptor' using git: 
\begin{lstlisting}[language=bash]
  $ git clone https://github.com/RescueOnWheels/Scriptor.git
\end{lstlisting}

Clone 'Rover' using git: 
\begin{lstlisting}[language=bash]
  $ git clone https://github.com/RescueOnWheels/Rover.git --recursive
\end{lstlisting}

Open the 'Scriptor' folder and start using the Rover: 
\begin{lstlisting}[language=bash]
  $ cd Scriptor && ./start.sh all
\end{lstlisting}