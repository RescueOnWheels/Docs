\section{Installation Guide}
This chapter gives a step by step tutorial to get the RSS up and running. After completing the installation, continue reading the usage guide.

\subsection{Headless setup procedure}

\begin{itemize}
    \item Step 1: Create an empty file. You can use Notepad on Windows or TextEdit to do so by creating a new file. Just name the file ssh. Save that empty file and dump it into boot partition (microSD).
    \item Step 2: Create another file name \textbf{wpa\_supplicant.conf}. This time you need to write a few lines of text for this file. For this file, you need to use the \textbf{FULL VERSION} of wpa\_supplicant.conf. Meaning you must have the 3 lines of data namely country, \textbf{ctrl\_interface} and \textbf{update\_config}.
\end{itemize}

\begin{lstlisting}
country="Your country"
ctrl_interface=DIR=/var/run/wpa_supplicant GROUP=netdev
update_config=1

network={
    ssid="your_real_wifi_ssid"
    scan_ssid=1
    psk="your_real_password"
    key_mgmt=WPA-PSK
}
\end{lstlisting}
\newpage
\subsection{Installation}
\noindent To install Rover and its dependencies from your RPI shell prompt, use the following command:
\begin{lstlisting}[language=bash]
  $ wget -q https://git.io/fx0ko -O /tmp/rover && bash /tmp/rover
\end{lstlisting}

\subsection{Manual Installation}

Update your system package list:
\begin{lstlisting}[language=bash]
  $ sudo apt-get -update -y
\end{lstlisting}

Upgrade all you installed packages to their latest version:
\begin{lstlisting}[language=bash]
  $ sudo apt-get dist-upgrade -y
\end{lstlisting}

To download and install newest version of Node.js, use the following command:
\begin{lstlisting}[language=bash]
  $ curl -sL https://deb.nodesource.com/setup_8.x | sudo -E bash -
\end{lstlisting}

Now install it by running: 
\begin{lstlisting}[language=bash]
  $ sudo apt-get install nodejs -y
\end{lstlisting}

To compile and install native addons from npm you also need to install build tools: 
\begin{lstlisting}[language=bash]
  $ sudo apt-get install build-essential -y
\end{lstlisting}

To download the Rover code you need to install git: 
\begin{lstlisting}[language=bash]
  $ sudo apt-get install git -y
\end{lstlisting}

Clone 'Scriptor' using git: 
\begin{lstlisting}[language=bash]
  $ git clone https://github.com/RescueOnWheels/Scriptor.git
\end{lstlisting}

Clone 'Rover' using git: 
\begin{lstlisting}[language=bash]
  $ git clone https://github.com/RescueOnWheels/Rover.git --recursive
\end{lstlisting}

\newpage
\subsection{UV4L}
The UV4L software suit consists of a series of highly configurable drivers, an optional Streaming Server module providing a RESTful API for custom development and various extensions for the server that cooperate together. The Streaming Server also provides the basic web UI for the end-users to try or use all the key functionalities directly. For maximum efficiency, each instance of UV4L runs as a single, independent system process which exploits the underlying hardware natively (whenever possible).

If you are running Raspbian Stretch instead, type:
\begin{lstlisting}[language=bash]
$ curl http://www.linux-projects.org/listing/uv4l_repo/lpkey.asc | sudo apt-key add -
\end{lstlisting}

and add the following line to the file /etc/apt/sources.list:
\begin{lstlisting}[language=bash]
$ deb http://www.linux-projects.org/listing/uv4l_repo/raspbian/stretch stretch main
\end{lstlisting}

Finally, we are ready to update the system and to fetch and install the packages:
\begin{lstlisting}[language=bash]
$ sudo apt-get update
$ sudo apt-get install uv4l uv4l-raspicam
\end{lstlisting}

Apart from the driver for the Raspberry Pi Camera Board, the following Streaming Server front-end and drivers can be optionally installed:
\begin{lstlisting}[language=bash]
$ sudo apt-get install uv4l-server uv4l-uvc uv4l-xscreen uv4l-mjpegstream uv4l-dummy uv4l-raspidisp
\end{lstlisting}
